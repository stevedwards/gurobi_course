
% Image courtesy of https://tex.stackexchange.com/questions/341949/how-to-plot-a-network-flow-with-tikz

\begin{figure}[H]
    \centering
    \begin{tikzpicture}[
      mycircle/.style={
         circle,
         draw=black,
         fill=gray,
         fill opacity = 0.3,
         text opacity=1,
         inner sep=0pt,
         minimum size=20pt,
         font=\small},
      myarrow/.style={-Stealth},
      node distance=0.6cm and 1.2cm
      ]
      \node[mycircle] (c1) {$s$};
      \node[mycircle,below right=of c1] (c2) {$v_2$};
      \node[mycircle,right=of c2] (c3) {$v_4$};
      \node[mycircle,above right=of c1] (c4) {$v_1$};
      \node[mycircle,right=of c4] (c5) {$v_3$};
      \node[mycircle,below right=of c5] (c6) {$t$};

    \foreach \i/\j/\txt/\p in {% start node/end node/text/position
      c1/c2/8/below,
      c1/c4/11/above,
      c2/c3/11/below,
      c3/c6/4/below,
      c4/c5/12/above,
      c5/c6/15/above,
      c5/c2/4/below,
      c3/c5/7/below,
      c2.70/c4.290/1/below}
       \draw [myarrow] (\i) -- node[sloped,font=\small,\p] {\txt} (\j);


     % draw this outside loop to get proper orientation of 10
     \draw [myarrow] (c4.250) -- node[sloped,font=\small,above,rotate=180] {10} (c2.110);
    \end{tikzpicture}
    \caption{Example network}
    \label{fig:my_label}
\end{figure}


